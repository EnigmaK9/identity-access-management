\documentclass[12pt, apa, man, natbib]{apa6}

\usepackage[utf8]{inputenc}
\usepackage{amsmath}
\usepackage{amsfonts}
\usepackage{amssymb}
\usepackage{graphicx}

\title{Storing Passwords}
\shorttitle{Password Security}
\author{Carlos Padilla}
\affiliation{Institute of Cybersecurity}
\date{\today}

\begin{document}

\maketitle

\section{Storing Passwords}
\subsection{Key Derivation Functions (KDF)}
\subsubsection{Lab 1: Password Dumps and Password Cracking}
This section covers the essential concepts of key derivation functions, password dumps, and password cracking.

\section{Specialized Attacks}
\subsection{Password Cracking Tools}
Various tools and techniques are used to crack passwords, highlighting the need for robust key derivation functions.

\section{Key Derivation Functions}
Key derivation functions are based on an irreversible hash function. A hash function takes an arbitrary amount of data as its input and uses it to calculate a fixed-length output value called a digest. The fascination of hashing the password lies in the fact that after hashing, the value is entirely unrecognizable, and the original input value cannot be retrieved.

\begin{quote}
    \textit{“To ensure the security of stored passwords, key derivation functions must be robust and resistant to attacks.”}
\end{quote}

\subsection{Hash Function}
A KDF implements a so-called difficulty factor. This value makes it intentionally more difficult to calculate the resulting hash to slow down any brute-force guessing attacks.

\section{What Determines the Strength of a Password Hash}
The strength of a password hash is determined by several factors:
\begin{itemize}
    \item Confusion
    \item Diffusion
    \item Hash collision
    \item Quality of key derivation function
    \item Password and derived key length
    \item Character set support
        \begin{itemize}
            \item 26 letters
            \item 52 characters
        \end{itemize}
    \item Password length
    \item Possible passwords
        \begin{itemize}
            \item Approximately 1 trillion
        \end{itemize}
    \item Key derivation function
\end{itemize}

\section{Key Derivation Functions (KDF)}
\subsection{MD5 Message Digest}
MD5 is a cryptographic hash algorithm that produces a hash value in hexadecimal format. This algorithm has serious weaknesses as it is known to have hash collisions. A collision happens when two unique plaintexts hash to the same hash value.

\subsection{Secure Hash Algorithm (SHA)}
SHA stands for Secure Hash Algorithm. SHA is a family of cryptographic functions with several iterations: SHA, SHA1, SHA2, SHA3.

\subsection{LAN Manager Hash}
The LAN Manager (LM) hash was introduced by Microsoft in 1980 and has multiple weaknesses. The LM hash is typically stored in the SAM/NTDS database system. Two severe weaknesses are a maximum password length of 14 characters and passwords being converted to uppercase.

\subsection{NT LAN Manager}
NT LAN Manager (NTLM) is the Microsoft authentication protocol created to be LM's successor.

\subsection{Lab 1: Hashing Basics}
This lab covers the basics of hashing and the importance of using secure hash functions.

\end{document}

